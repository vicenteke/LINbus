\begin{document}

Logo, foi feita uma implementação do protocolo \textit{Local Interconnect Network} (LIN) e avaliou-se o seu desempenho por métodos gráficos e estatísticos. O experimento consistiu na comunicação entre dois microcontroladores, um executando a função de mestre e outro de escravo. Um terceiro microcontrolador foi adicionado para medir os níveis de tensão na linha, e um gráfico animado, simulando a execução em tempo real, foi gerado a partir dos dados obtidos. Uma fonte de ruído externo foi adicionada, fornecendo uma análise mais completa do protocolo. Além disso, um método de detecção de erros extra foi adicionada ao sistema, o CRC, complementando os já utilizados \textit{checksum} e bits de paridade.

Apesar de ser tido como uma alternativa mais barata e, contudo, menos confiável de protocolo de comunicação serial para redes automotivas, os experimentos mostraram um ótimo desempenho do LIN. A ocorrência de valores fora dos limites aceitáveis é desprezível, com uma estabilidade visível nos níveis de tensão para ambos níveis lógicos. Entretanto, o protótipo foi feito com apenas um escravo e com uma linha extremamente curta. Também, o ruído eletromagnético em veículos pode ser maior em situações práticas do que aquele adicionado. Trabalhos futuros podem reconsiderar esses aspectos durante a análise estatística dos dados, estudando seus efeitos.

Portanto, o LIN é um protocolo de comunicação feito para redes automotivas e que expõe ótimos resultados experimentais, devendo ser considerados como uma alternativa viável para muitos dos subsistemas que compõem as funcionalidades veiculares.

\end{document}